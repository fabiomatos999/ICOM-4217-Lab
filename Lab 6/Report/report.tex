\documentclass[journal]{IEEEtran}
\hyphenation{op-tical net-works semi-conduc-tor}
\usepackage{siunitx}
\usepackage{graphicx}
\usepackage{listings}
\usepackage{hyperref}
\usepackage{float}
\usepackage{tabularx}
\usepackage{unicode}
\usepackage[table]{xcolor}
\newcommand\createfigure[2]{
  \begin{figure}[H]
    \centering \includegraphics[width=0.4\textwidth]{#1}
    \caption{#2}
  \end{figure}}
\begin{document}
\title{Experiment 6 Tasks\\ Lab Report} \author{Fabio J. Matos
  Nieves (802-18-6134), Enrique Chompre Gonzalez (802-18-6106), \\Guillermo Colón Bernardi (802-16-1272)}
\maketitle
\begin{abstract}
  The purpose of this document is to detail an implementation of motor DC, Servo and Stepper motors using a MSP430 Microcontroller, a half H-Bridge driver and darlington transistor array.
\end{abstract}
\begin{IEEEkeywords}
  MCU, Motor Driver, H-Bridge, Darlington Transistor Array
\end{IEEEkeywords}
\IEEEpeerreviewmaketitle
\section{Introduction}
\IEEEPARstart{T}{he} first task for this laboratory assignment consists driving a DC motor using a half H-Bridge Driver. A DC motor is a motor that only has positive and negative terminals and runs at maximum speed when the voltage is at the recommended maximum. An H-Bridge driver is a set of complementary transistorsthat are driven by 2 pins on a microcontroller that allow for the reverse and forward current flow within a motor which allows a DC motor to move clockwise or counterclockwise depending on which set of transistors are on or off at any given time. The second task for this assignment was exactly the same of the first but using an IC H-Bridge driver for the motor instead of the through hole transistor array. The third task of this assignment was to drive a servo motor at various angles. A servo motor is type of motor with 3 pins going to it, VDD, ground and control. The control pin is used to send a 50\si{\Hz} PWM signal with a duty cycle between 5\% and 12.5\% going from \ang{-90} to \ang{90}. The fourth assignment consists of driving a stepper motor using a Darlington Transistor Array.
\section{Materials and Methods}
Materials:
\section{Results}
\nocite{rojasEmbeddedSystemsDesign2016}
\nocite{LCDControllerDatasheets}
\bibliographystyle{IEEEtran}
\bibliography{./references.bib}
\end{document}
