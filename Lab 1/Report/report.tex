\documentclass[journal]{IEEEtran} \hyphenation{op-tical net-works
  semi-conduc-tor} \usepackage{textgreek} \usepackage{graphicx}
\usepackage{listings} \usepackage{hyperref}
\begin{document}
\title{Experiment 2 Complementary Tasks\\ Lab Report} \author{Fabio J. Matos
  Nieves, Enrique Chompre Gonzalez}
\maketitle
\begin{abstract}
  The purpose of this document is to detail our implementation of the
  complementary tasks from the second experiment of the ICOM 4217 Laboratory.
\end{abstract}
\begin{IEEEkeywords}
  MCU, LDC, Display, Pins,
\end{IEEEkeywords}
\IEEEpeerreviewmaketitle
\section{Introduction}
\IEEEPARstart{T}{he} complementary tasks for this laboratory assignment consist
of using push buttons to make an LCD panel scroll through a list of 16 messages
either up or down through the list. To accomplish this task we first had to
initialize the LCD, and make several functions such as clearing the display,
moving the cursor to various positions, printing out a character and printing
out a message\cite{rojasEmbeddedSystemsDesign2016}.
\section{Materials and Methods}
List of Materials
\begin{itemize}
  \item 1 Development Board
  \item 1 IDE Application
  \item 1 LCD Display: 2 lines 16 characters W/HD 44780
        Controller\cite{LCDControllerDatasheets}
  \item 2 Light Emitting Diodes
  \item 2 330\textOmega \space1/4 W Carbon Fill Resistor
  \item 2 4.7K\textOmega \space1/4 W Carbon Fill Resistor
  \item 2 Momentary Switch
\end{itemize}
The software plan for the complementary task is as follows:
\begin{enumerate}
  \item Initialize the LCD
  \item Allocate memory for the array of strings.
  \item Display the first two strings of the array.
  \item Poll both push buttons in an infinite loop.
  \item Display new elements of the list of strings depending on which
        pushbutton is being pressed.
\end{enumerate}
\vspace{7cm} Wiring Diagram:
\begin{figure}[ht]
  \centering \includegraphics[width=0.4\textwidth]{./Figures/wiring-diagram.jpg}
  \caption{Wiring Diagram for LCD display and MCU}
\end{figure}
\section{Results}
The source code is located on GitHub
\href{https://github.com/fabiomatos999/ICOM-4217-Lab/blob/main/Lab\%201/1.2.2.c}{here}.
\bibliographystyle{IEEEtran} \bibliography{../../My Library}
\end{document}
